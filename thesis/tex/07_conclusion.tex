\chapter{Conclusion}
\label{chapter:conclusion}

In \cref{chapter:introduction} of this thesis, the need for verification in complex software projects was discussed. Modern approaches to software verification use models to verify a piece of software. However, not all models are suited for this task. Moreover, creating multiple models in different modelling languages for the same software is time-consuming and expensive. The use of model transformations to automatically transform models between different modelling languages was presented as a solution. However, in order for model transformations to be of use in the context of software verification, a formal foundation for the model transformations is required. This thesis has shown a formal foundation for model transformations between EMF/Ecore and GROOVE, which can be of use in the field of software verification.

As discussed in \cref{sec:background:eclipse_modeling_framework}, EMF/Ecore is a popular modelling language for modelling software. However, its models are not well-suited for verification. As discussed in \cref{sec:background:groove}, GROOVE is created especially for software verification but uses a different modelling language than EMF/Ecore. Therefore, model transformations should be used to convert between the two models automatically. A formalisation of the modelling languages themselves was provided in \cref{chapter:formalisations}. These formalisations are the foundation of the formalisation of the model transformations presented as part of the transformation framework.

The transformation framework presented in \cref{chapter:transformation_framework} provides the main result of this thesis. The transformation framework includes a formalisation of transformation functions between Ecore models and GROOVE graphs, which allows the user to reason about these transformation functions formally. Furthermore, the transformation framework has presented a structured way to compose these transformation functions iteratively while maintaining the correctness. The composability of these transformation functions is essential, as it allows the user of the framework to build significant model transformations without loss of correctness. The correctness property is relevant in the field of software verification, as it allows for verification of Ecore models within GROOVE, without the loss of confidence that the results might be incorrect due to transformation errors.

In order to further validate the transformation framework, \cref{chapter:library_of_transformations} has presented a small library of transformations that can be used within the transformation framework. The transformations from the library presented in this chapter are all small transformations with a few elements, that can be added to a larger model using the framework. In this way, the transformations can be used to compose significant transformations. The library of transformations allows a user to build specific models without the need to define transformation functions for each created model.

Finally, \cref{chapter:application} has shown an application of the transformation framework and the library of transformations in a practical example. Throughout the chapter, a single model is built from scratch, showing each of the steps taken to build the model using the transformation framework. Each step combines a transformation from the library of transformations with the existing transformation function, making the transformation function larger and allowing more complex models to be transformed.

In this final chapter, the work presented by the thesis will be concluded. First, the advantages and limitations of the work will be discussed. Then the work will be evaluated and the research questions introduced in \cref{sec:introduction:research_question} will be answered. Finally, some proposals for future work are discussed.

\section{Advantages \& Limitations}
\label{sec:conclusion:advantages_and_limitations}

Although the approach taken by this thesis has some distinct advantages over other possible methods for proving the correctness of model transformations, there are also some limitations of the work that need to be discussed. These advantages and limitations are further discussed in this section before the work is evaluated.

As explained earlier, the transformation framework presented in \cref{chapter:transformation_framework} is considered the main result of this thesis. This framework is compelling, in that it allows to compose model transformations, which allows for creating possibly infinitely large model transformations between Ecore and GROOVE. In order to have composable model transformations, the concept of combining models and graphs is used. Within this work, it has been chosen to maintain the correctness of the transformation at each step. This correctness means that only valid or consistent models and graphs are used within each step. The use of correct models and graphs is favourable because it makes proving the correctness of the combination much more straightforward. The correctness properties of the individual models and graphs can be used to prove the correctness of the combinations.

Maintaining correctness in each step of the composition is a definite advantage to maintaining the proof of correctness. However, it also presents limitations for the transformation steps. Because of the required correctness properties, each transformation step has to be valid itself. For some transformations, this means quite significant transformation steps. For example, when introducing a new field on the type level, it is required to introduce the value for this field for all related objects on the instance level within one step. The introduction of all values at once is the only way that correctness is maintained. However, these are already quite large proofs, and they become increasingly complicated when adding inheritance. If a field is added to a supertype, a value for the field must be introduced for all instances of the supertype and its subtypes. Introducing values in this way means that an even more significant transformation step is needed to achieve such a composition. In practice, it is possible to use more substantial transformations, but it means that the complexity of proving each transformation step is increased.

The consequences of this limitation are directly visible from the library of transformations, \cref{chapter:library_of_transformations}. The transformations that introduce new fields are already quite complex, especially in their proofs. Furthermore, these transformations cannot be used on extended types because of the limitation above. Separate transformations need to be proven to allow the addition of fields to an extended type. 

Because of the limitations of the transformation framework and the limited amount of time available for this thesis, the library of transformations is quite small and incomplete. Only a selected set of transformations is presented here, which does not even cover all concepts of Ecore. On the side of Ecore, the following concepts still need to be covered:
\begin{itemize}
    \item The introduction of fields on types that are extended by other types, as discussed above.
    \item The introduction of fields typed by different container types still needs to be covered. Only one transformation shows the use of a $\type{setof}$-type in conjunction with a $\type{containment}$ property (\cref{subsec:library_of_transformations:type_level_transformations:contained_class_set_fields} and \cref{subsec:library_of_transformations:instance_level_transformations:contained_class_set_field_values}), but the other containers also need to be covered. Also, container types containing attributes still need to be covered.
    \item The concept of multiple inheritance, which is supported by Ecore and GROOVE, but not used in any transformation. Only one transformation with `single' inheritance is shown as part of \cref{subsec:library_of_transformations:type_level_transformations:regular_subclasses} and \cref{subsec:library_of_transformations:instance_level_transformations:objects_of_subtype}.
    \item Most of the different model properties (\cref{defin:formalisations:ecore_formalisation:type_models:type_model_properties}), $\type{defaultValue}$, $\type{identity}$, $\type{keyset}$, $\type{opposite}$ and $\type{readonly}$ to be precise, are not yet covered by any of the transformations. The model properties that are covered, $\type{abstract}$ and $\type{containment}$ are not yet covered in their full potential.
    \item The introduction of constants and their corresponding values is not covered yet. Since they are only used in conjunction with $\type{defaultValue}$ properties, it makes sense to cover them at the same time.
\end{itemize}
Covering all concepts of Ecore with one or more encodings in GROOVE would mean that all concepts of GROOVE are also covered. However, this way of achieving coverage means the addition of a lot more transformations, which could be its own research. Therefore, this is considered future work as described in \cref{sec:conclusion:future_work}.

A different limitation of this thesis, in general, is the focus on syntactical correctness only. No effort has been made to prove the correctness of the semantics of a model under transformation. This correctness property has been excluded on purpose, as EMF/Ecore is a quite general modelling framework in which a lot of different software models can be expressed. Therefore, it is difficult to prove something about the semantics on an abstract level. A consequence of this decision is that curious encodings are possible, which are still syntactically correct. For example, one might create an encoding that multiplies all integer values within a model with a certain number $x$, when transformed into a GROOVE graph. Then, a different transformations function can be used to convert back to a model, dividing all integer values with the same number $x$. Although this is syntactically correct, it could have enormous implications for the use within software verification, as the values of the model have changed. If this is not taken into account beforehand, the results of the software verification could still be questionable.

Although the work presented by this thesis has some limitations, the work is still considered a useful contribution. It is believed (although not proven) that it is possible to work around the limitations of the transformation framework, possibly with more substantial transformations. Moreover, the semantics of a transformation could be addressed in future research and does not invalidate the work presented here. Therefore, the work presented in this thesis should be used as a foundation, rather than a piece of work that is ready to use.
\section{Evaluation}
\label{sec:conclusion:evaluation}

This section will evaluate the research carried out in this thesis and answer the main research question presented in \cref{sec:introduction:research_question}. In order to answer the main research question, the subquestions need to be evaluated and answered. The answers to the subquestions are the following:

\begin{enumerate}
    \item ``What is a suitable formalisation of Ecore models and what Ecore models are valid within this formalisation?'' 
    
    For this thesis, a suitable formalisation of Ecore was presented in \cref{sec:formalisations:ecore_formalisation}. The formalisation is considered suitable since it can express almost all concepts of Ecore. Therefore, it is also able to express all the models used throughout this thesis, which are type models and instance models.
    
    The set of valid models within this formalisation depend on the model level. For type models, the set of consistent models is constrained by \cref{defin:formalisations:ecore_formalisation:type_models:type_model_consistency}. For instance models, the set of valid models is constrained by \cref{defin:formalisations:ecore_formalisation:instance_models:model_validity}. These definitions answer the question which models are valid within the formalisation, and the examples shown throughout this thesis show the existence of these models.
    
    The answer to this question is validated using the existing theory available on Ecore. Also, each part of the formalisation can be traced back to elements within the Ecore metamodel. Together, the answer to the first question is considered validated.
    
    \item ``What is a suitable formalisation of GROOVE grammars and what GROOVE grammars are valid within this formalisation?'' 
    
    Just like the previous question, a suitable formalisation was presented within this thesis. The suitable formalisation for GROOVE graphs can be found in \cref{sec:formalisations:groove_formalisation}. The formalisation is considered suitable since it can express the relevant GROOVE graphs used throughout this thesis, which are type models and instance models.
    
    The set of valid grammars within the formalisation depend on the set of graphs defined for a GROOVE grammar. In order to answer the main research question, it is only relevant to know when type graphs and instance graphs are valid. For type graphs, the set of consistent graphs is constrained by \cref{defin:formalisations:groove_formalisation:type_graphs:type_graph_validity}. For instance graphs, the set of valid graphs is constrained by \cref{defin:formalisations:groove_formalisation:instance_graphs:instance_graph_validity}. These definitions answer the question of which graphs are valid within the formalisation, and the examples shown throughout this thesis show the existence of these graphs. Since the main research question only needs to show the validity of the presented graph types, the answer to the question is sufficient.
    
    Once more, the answer to this question is validated using the existing theory available. Existing theory on GROOVE and graph theory has been used to validate the formalisation, and therefore the answer to this question is considered validated.
    
    \item ``What is a suitable formalisation for the model transformations between Ecore and GROOVE?'' 
    
    A formalisation for model transformations between Ecore and GROOVE was given as part of the presented transformation framework in \cref{chapter:transformation_framework}. \cref{defin:transformation_framework:type_models_and_type_graphs:combining_transformation_functions:transformation_function_type_model_type_graph}, \cref{defin:transformation_framework:type_models_and_type_graphs:combining_transformation_functions:transformation_function_type_graph_type_model}, \cref{defin:transformation_framework:instance_models_and_instance_graphs:combining_transformation_functions:transformation_function_instance_model_instance_graph} and \cref{defin:transformation_framework:instance_models_and_instance_graphs:combining_transformation_functions:transformation_function_instance_graph_instance_model} are the relevant definitions that specify the properties of (composable) transformation functions between different models and graphs.
    
    The formalisation is considered suitable since it gives rise to a significant set of possible transformations, which allow for building large complex models. The existence and correctness of this set are validated using the library of transformations in \cref{chapter:library_of_transformations}, which defines and proves a small set of possible transformations within this formalisation. These transformations show the existence of model transformations between Ecore and GROOVE that fit within the presented formalisation. Furthermore, the transformations can map between the formalisations of Ecore and GROOVE by using these formalisations as input and output, giving confidence in the fact that multiple transformation functions can cover all elements of Ecore and GROOVE. Therefore, the formalisation of the model transformations themselves can be considered suitable.
    
    \item ``What model transformations are correct within the formalisation?''
    
    \cref{defin:transformation_framework:type_models_and_type_graphs:combining_transformation_functions:transformation_function_type_model_type_graph}, \cref{defin:transformation_framework:type_models_and_type_graphs:combining_transformation_functions:transformation_function_type_graph_type_model}, \cref{defin:transformation_framework:instance_models_and_instance_graphs:combining_transformation_functions:transformation_function_instance_model_instance_graph} and \cref{defin:transformation_framework:instance_models_and_instance_graphs:combining_transformation_functions:transformation_function_instance_graph_instance_model} constrain the transformation functions that are considered syntactically valid within the formalisation. Furthermore, the library of transformations in \cref{chapter:library_of_transformations} has shown some examples of transformations that are proven to be valid with respect to these definitions, and each of these transformations is also reversible. Therefore, multiple transformations have been presented which are each syntactically correct within the formalisation. Furthermore, the formalisation allows us the combine these transformation functions (as presented in \cref{chapter:transformation_framework}), while maintaining the correctness of the functions. This has also been proven as part of this thesis.
    
    The answer to this question is validated by validating all the correctness proofs. For this thesis, all the proofs are validated using the Isabelle theorem prover, as presented by \cref{sec:background:theorem_proving_using_isabelle}. Therefore, it is validated that the proofs are correct, meaning that by following the definitions, it is clear which model transformations are syntactically valid within the formalisation. Therefore, the question is considered answered.
    
    \item ``How can correct model transformations between Ecore and GROOVE be composed?''
    
    As explained earlier in \cref{sec:introduction:validation}, answering this question consists of two parts. First \cref{chapter:transformation_framework} has shown on a formal level how model transformations are combined. This formalisation has been done for transformations at both levels. Secondly, it has been shown how to compose these transformations in practice, which can be found in \cref{chapter:application}.
    
    \cref{chapter:transformation_framework} describes the composability of model transformations by explaining the necessary definitions to combine models, graphs and transformation functions. Using \cref{defin:transformation_framework:type_models_and_type_graphs:combining_transformation_functions:tg_combine_mapping_function_correct}, \cref{defin:transformation_framework:type_models_and_type_graphs:combining_transformation_functions:tmod_combine_mapping_function_correct}, \cref{defin:transformation_framework:type_models_and_type_graphs:combining_transformation_functions:tmod_combine_mapping_function_correct} and \cref{defin:transformation_framework:instance_models_and_instance_graphs:combining_transformation_functions:imod_combine_mapping_function_correct} it is possible to show that these definitions indeed give rise to a new transformation function, which is then proven to be syntactically correct.
    
    \cref{chapter:application} shows how to apply these definitions by composing a large transformation function out of the small transformation functions presented in the library of transformations (see \cref{chapter:library_of_transformations}). This example shows how the definitions from \cref{chapter:transformation_framework} can be applied in practice, answering the practical part of the question.
    
    As explained before, all proofs within this thesis have been validated using the Isabelle theorem prover, as presented by \cref{sec:background:theorem_proving_using_isabelle}. That means that the proofs presented to prove the composition of transformation functions are correct. The correctness of these proofs validates that the method of composing transformation functions is syntactically correct. Furthermore, the example of \cref{chapter:application} shows that it is possible to apply these definitions in practice, which validates that there is actual use for the composability definitions. Together this shows that all parts of the answer are validated, so the question is considered validated.
\end{enumerate}

With the answers to the subquestions given, it is possible to answer the main research question:

``What is a suitable formalisation for composable model transformations between Ecore and GROOVE that gives rise to correct model transformations between Ecore and GROOVE?''

In order to give a formalisation for the model transformations, formalisations for Ecore and GROOVE were needed. Subquestions 1 and 2 have shown suitable transformations for both languages. The formalisation for composable model transformations is given as part of \cref{chapter:transformation_framework}. Subquestions 3 explains that the formalisation is suitable, whereas subquestion 5 shows that the formalisation indeed allows for composable model transformations. The correctness of these transformations is proven, which is explained in subquestions 4 and 5. Finally, \cref{chapter:library_of_transformations} and \cref{chapter:application} have shown that there indeed exist such model transformations in practice. The existence of these model transformations is also explained in subquestions 3 and 5. Altogether, these answer the main research question, as the thesis has shown that there exists a suitable formalisation for composable model transformations between Ecore and GROOVE. It has also shown that this formalisation gives rise to syntactically correct model transformations between Ecore and GROOVE.
\section{Future work}
\label{sec:conclusion:future_work}

Although this thesis has provided a complete framework for composing transformation functions between Ecore and GROOVE, much work remains to be done to make this solution viable in practice. In future work, limitations of this work could be addressed, and additional work can be performed in order to make implementations of the work possible. Possible improvements and future work are discussed in this section.


\subsection{Improvements to the transformation framework}
\label{subsec:conclusion:future_work:improvements_to_the_transformation_framework}

This thesis has provided a transformation framework for composable model transformations between Ecore and GROOVE. As discussed in \cref{sec:conclusion:advantages_and_limitations}, the choices made within this transformation framework have advantages and limitations. An important limitation is the requirement to maintain correctness at each step of composing a transformation function. As a consequence, the small transformations presented in the library of transformations can become quite substantial, and are more difficult to prove.

One could investigate the possibility of making the requirement of correctness less strict. This could be done by allowing some partial compositions, that need to be applied in a specific order, to achieve once more a correct transformation. These partial compositions would make the individual transformation steps smaller, and thus easier to prove. However, it will probably introduce some new requirements to maintain the general correctness of the transformation, which should be researched further.


\subsection{Complete the library of transformations}
\label{subsec:conclusion:future_work:extend_the_library_of_transformations}

This thesis has provided a non-exhaustive library of transformations that can be applied within the transformation framework. As explained in \cref{sec:conclusion:advantages_and_limitations}, several concepts of Ecore are not yet covered. Future research might focus on defining a complete library of transformations, that can compose every possible transformation function. Such research would consist of adding transformations for missing concepts and possibly a proof that all valid models and graphs can indeed be built.

In order to achieve completeness of the library, at least the following transformations should be provided:
\begin{itemize}
    \item For all transformations that introduce fields, there should be counterparts that can introduce these fields on types that are extended by other types. In order to create these transformations, the set of shared objects should contain not only all instances of the supertype but also all instances of all subtypes. Furthermore, for all these objects, a value needs to be introduced for the newly introduced field. Please note that this should be done for both the introduction of fields on abstract classes and the introduction of fields on regular classes. When done carefully, the proof created for the instance level could be reused for both of these transformations. Furthermore, the proofs created for regular classes might be able to replace the existing transformations for introducing fields.
    
    \item New transformations need to be introduced that cover all the possible container types. This means that transformations should be created that introduce a field typed by a $\type{bagof}$, $\type{setof}$, $\type{seqof}$ or $\type{ordof}$ container type, containing any other possible type. Because of the limitations of the transformation framework, it will not be possible to prove this for all contained types at once, so multiple transformations will be needed here.
    
    \item A transformation should be created that allows for introducing multiple inheritance. Multiple inheritance can be introduced by either creating a transformation that introduces a new type that extends from a set of existing types, or by creating a transformation that introduces a set of new types from which one existing subtype does inherit. The new types should be created in one transformation, to be able to proof that the inheritance relation remains valid.
    
    \item For all transformations introducing some field, an additional transformation would need to be created that introduces a $\type{defaultValue}$ property on the corresponding field. This transformation would also need to introduce a new constant and its value.
    
    \item A transformation should be created that introduces a new class type with an $\type{identity}$ property. Because of the limitations of the transformation framework, all the attributes that are part of the identity property need to be introduced as well. Furthermore, all the instances of the new class type need to be created within the same transformation, and all values for all the attributes should be defined. Such a transformation would be substantial, and would undoubtedly benefit from the possibility to do partial compositions, as proposed by \cref{subsec:conclusion:future_work:improvements_to_the_transformation_framework}.
    
    \item Multiple transformations should be created which introduce a new field with an $\type{keyset}$ property. Because of the limitations of the transformation framework, all the attributes that are part of the keyset property need to be introduced as well. Furthermore, a value for the field and all related attributes needs to be introduced at the instance level, within one transformation step. Once more, such a transformation would be substantial, and would undoubtedly benefit from the possibility to do partial compositions, as proposed by \cref{subsec:conclusion:future_work:improvements_to_the_transformation_framework}.
    
    \item Multiple transformations should be created which introduce a new field with an $\type{opposite}$ property. Introducing the $\type{opposite}$ property can be done between existing types and instances, it only has to be made sure that transformations exist for all possible types for a field with an $\type{opposite}$ property. Extra care should be given to introducing a new field that has a $\type{opposite}$ property and a $\type{containment}$ property on one of the fields. In this case, it is needed to introduce the instances contained by the containment relation as well.
    
    \item For all the transformations introducing some field, an additional transformation would need to be created that introduces a $\type{readonly}$ property on the corresponding field. Since a read-only property is only used for the semantics of the model and not anywhere in the syntax, this should be quite trivial.
    
    \item Transformations need to be created for some combinations of properties, especially $\type{containment}$ and $\type{identity}$, to ensure that models combining these properties can be created within the transformation framework.
\end{itemize}
Although it is believed that introducing all these transformations would eventually result in a complete library of transformations, no proof for this claim exists. Therefore, this should be taken into account when performing this work.


\subsection{Add more encodings}
\label{subsec:conclusion:future_work:add_more_encodings}

This thesis has shown that elements in Ecore can have multiple encodings in GROOVE. An example of this is given in \cref{subsec:library_of_transformations:type_level_transformations:enumeration_types} and \cref{subsec:library_of_transformations:instance_level_transformations:enumeration_values}. Future work might focus on defining more encodings for different elements of Ecore. Having the choice between multiple encodings for the same element when transforming an Ecore model to GROOVE might be beneficial for the verification of specific properties. Different encodings might have their advantages and disadvantages within the field of verification, and therefore a choice between encodings might make more use cases possible.

Each of these encodings should be proven correct within the transformation framework. Moreover, as shown in \cref{subsec:library_of_transformations:type_level_transformations:enum_fields} and \cref{subsec:library_of_transformations:instance_level_transformations:enum_field_values}, it might be the case that introducing new encodings might also need specific transformations for related elements that use the encoding. Therefore, providing more encodings could be a quite substantial amount of work.

Besides introducing more encodings, such research would also include research on the possible encodings for an element. Furthermore, the practical use of each of these encodings could be investigated.


\subsection{Implementation}
\label{subsec:conclusion:future_work:implementation}

The framework presented in this thesis only represents a mathematical foundation. In order to apply this framework in practice, it needs to be implemented as part of EMF, GROOVE or as a standalone tool. Such implementation would allow a user to automatically transform Ecore models into GROOVE graphs and vice versa while having the framework as a formal foundation to show the syntactical correctness of the transformations.

The implementation of this framework is not a trivial task in itself. Such an implementation would need to figure out how to decompose models and graphs into their separate components in order to build a transformation. This decomposition might give rise to more fundamental research, as the current framework does not describe the decomposition of models and graphs. Furthermore, an approach to the decomposition of models and graphs might be an ambiguous process, meaning that a model or graph can be decomposed in more than one way. Research on how to handle this ambiguity also needs to be carried out when applicable. Furthermore, an implementation would need to deal with multiple possible encodings when composing the transformation function, as it is possible to transform a model using multiple correct graph encodings.