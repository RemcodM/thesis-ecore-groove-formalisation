\section{Evaluation}
\label{sec:conclusion:evaluation}

This section will evaluate the research carried out in this thesis and answer the main research question presented in \cref{sec:introduction:research_question}. In order to answer the main research question, the subquestions need to be evaluated and answered. The answers to the subquestions are the following:

\begin{enumerate}
    \item ``What is a suitable formalisation of Ecore models and what Ecore models are valid within this formalisation?'' 
    
    For this thesis, a suitable formalisation of Ecore was presented in \cref{sec:formalisations:ecore_formalisation}. The formalisation is considered suitable since it can express almost all concepts of Ecore. Therefore, it is also able to express all the models used throughout this thesis, which are type models and instance models.
    
    The set of valid models within this formalisation depend on the model level. For type models, the set of consistent models is constrained by \cref{defin:formalisations:ecore_formalisation:type_models:type_model_consistency}. For instance models, the set of valid models is constrained by \cref{defin:formalisations:ecore_formalisation:instance_models:model_validity}. These definitions answer the question which models are valid within the formalisation, and the examples shown throughout this thesis show the existence of these models.
    
    The answer to this question is validated using the existing theory available on Ecore. Also, each part of the formalisation can be traced back to elements within the Ecore metamodel. Together, the answer to the first question is considered validated.
    
    \item ``What is a suitable formalisation of GROOVE grammars and what GROOVE grammars are valid within this formalisation?'' 
    
    Just like the previous question, a suitable formalisation was presented within this thesis. The suitable formalisation for GROOVE graphs can be found in \cref{sec:formalisations:groove_formalisation}. The formalisation is considered suitable since it can express the relevant GROOVE graphs used throughout this thesis, which are type models and instance models.
    
    The set of valid grammars within the formalisation depend on the set of graphs defined for a GROOVE grammar. In order to answer the main research question, it is only relevant to know when type graphs and instance graphs are valid. For type graphs, the set of consistent graphs is constrained by \cref{defin:formalisations:groove_formalisation:type_graphs:type_graph_validity}. For instance graphs, the set of valid graphs is constrained by \cref{defin:formalisations:groove_formalisation:instance_graphs:instance_graph_validity}. These definitions answer the question of which graphs are valid within the formalisation, and the examples shown throughout this thesis show the existence of these graphs. Since the main research question only needs to show the validity of the presented graph types, the answer to the question is sufficient.
    
    Once more, the answer to this question is validated using the existing theory available. Existing theory on GROOVE and graph theory has been used to validate the formalisation, and therefore the answer to this question is considered validated.
    
    \item ``What is a suitable formalisation for the model transformations between Ecore and GROOVE?'' 
    
    A formalisation for model transformations between Ecore and GROOVE was given as part of the presented transformation framework in \cref{chapter:transformation_framework}. \cref{defin:transformation_framework:type_models_and_type_graphs:combining_transformation_functions:transformation_function_type_model_type_graph}, \cref{defin:transformation_framework:type_models_and_type_graphs:combining_transformation_functions:transformation_function_type_graph_type_model}, \cref{defin:transformation_framework:instance_models_and_instance_graphs:combining_transformation_functions:transformation_function_instance_model_instance_graph} and \cref{defin:transformation_framework:instance_models_and_instance_graphs:combining_transformation_functions:transformation_function_instance_graph_instance_model} are the relevant definitions that specify the properties of (composable) transformation functions between different models and graphs.
    
    The formalisation is considered suitable since it gives rise to a significant set of possible transformations, which allow for building large complex models. The existence and correctness of this set are validated using the library of transformations in \cref{chapter:library_of_transformations}, which defines and proves a small set of possible transformations within this formalisation. These transformations show the existence of model transformations between Ecore and GROOVE that fit within the presented formalisation. Furthermore, the transformations can map between the formalisations of Ecore and GROOVE by using these formalisations as input and output, giving confidence in the fact that multiple transformation functions can cover all elements of Ecore and GROOVE. Therefore, the formalisation of the model transformations themselves can be considered suitable.
    
    \item ``What model transformations are correct within the formalisation?''
    
    \cref{defin:transformation_framework:type_models_and_type_graphs:combining_transformation_functions:transformation_function_type_model_type_graph}, \cref{defin:transformation_framework:type_models_and_type_graphs:combining_transformation_functions:transformation_function_type_graph_type_model}, \cref{defin:transformation_framework:instance_models_and_instance_graphs:combining_transformation_functions:transformation_function_instance_model_instance_graph} and \cref{defin:transformation_framework:instance_models_and_instance_graphs:combining_transformation_functions:transformation_function_instance_graph_instance_model} constrain the transformation functions that are considered syntactically valid within the formalisation. Furthermore, the library of transformations in \cref{chapter:library_of_transformations} has shown some examples of transformations that are proven to be valid with respect to these definitions, and each of these transformations is also reversible. Therefore, multiple transformations have been presented which are each syntactically correct within the formalisation. Furthermore, the formalisation allows us the combine these transformation functions (as presented in \cref{chapter:transformation_framework}), while maintaining the correctness of the functions. This has also been proven as part of this thesis.
    
    The answer to this question is validated by validating all the correctness proofs. For this thesis, all the proofs are validated using the Isabelle theorem prover, as presented by \cref{sec:background:theorem_proving_using_isabelle}. Therefore, it is validated that the proofs are correct, meaning that by following the definitions, it is clear which model transformations are syntactically valid within the formalisation. Therefore, the question is considered answered.
    
    \item ``How can correct model transformations between Ecore and GROOVE be composed?''
    
    As explained earlier in \cref{sec:introduction:validation}, answering this question consists of two parts. First \cref{chapter:transformation_framework} has shown on a formal level how model transformations are combined. This formalisation has been done for transformations at both levels. Secondly, it has been shown how to compose these transformations in practice, which can be found in \cref{chapter:application}.
    
    \cref{chapter:transformation_framework} describes the composability of model transformations by explaining the necessary definitions to combine models, graphs and transformation functions. Using \cref{defin:transformation_framework:type_models_and_type_graphs:combining_transformation_functions:tg_combine_mapping_function_correct}, \cref{defin:transformation_framework:type_models_and_type_graphs:combining_transformation_functions:tmod_combine_mapping_function_correct}, \cref{defin:transformation_framework:type_models_and_type_graphs:combining_transformation_functions:tmod_combine_mapping_function_correct} and \cref{defin:transformation_framework:instance_models_and_instance_graphs:combining_transformation_functions:imod_combine_mapping_function_correct} it is possible to show that these definitions indeed give rise to a new transformation function, which is then proven to be syntactically correct.
    
    \cref{chapter:application} shows how to apply these definitions by composing a large transformation function out of the small transformation functions presented in the library of transformations (see \cref{chapter:library_of_transformations}). This example shows how the definitions from \cref{chapter:transformation_framework} can be applied in practice, answering the practical part of the question.
    
    As explained before, all proofs within this thesis have been validated using the Isabelle theorem prover, as presented by \cref{sec:background:theorem_proving_using_isabelle}. That means that the proofs presented to prove the composition of transformation functions are correct. The correctness of these proofs validates that the method of composing transformation functions is syntactically correct. Furthermore, the example of \cref{chapter:application} shows that it is possible to apply these definitions in practice, which validates that there is actual use for the composability definitions. Together this shows that all parts of the answer are validated, so the question is considered validated.
\end{enumerate}

With the answers to the subquestions given, it is possible to answer the main research question:

``What is a suitable formalisation for composable model transformations between Ecore and GROOVE that gives rise to correct model transformations between Ecore and GROOVE?''

In order to give a formalisation for the model transformations, formalisations for Ecore and GROOVE were needed. Subquestions 1 and 2 have shown suitable transformations for both languages. The formalisation for composable model transformations is given as part of \cref{chapter:transformation_framework}. Subquestions 3 explains that the formalisation is suitable, whereas subquestion 5 shows that the formalisation indeed allows for composable model transformations. The correctness of these transformations is proven, which is explained in subquestions 4 and 5. Finally, \cref{chapter:library_of_transformations} and \cref{chapter:application} have shown that there indeed exist such model transformations in practice. The existence of these model transformations is also explained in subquestions 3 and 5. Altogether, these answer the main research question, as the thesis has shown that there exists a suitable formalisation for composable model transformations between Ecore and GROOVE. It has also shown that this formalisation gives rise to syntactically correct model transformations between Ecore and GROOVE.