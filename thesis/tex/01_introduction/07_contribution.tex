\section{Contribution}
\label{sec:introduction:contribution}

This section discusses the intended contribution of this thesis to the active field of research. This thesis will propose a transformation framework for bidirectional transformations between EMF/Ecore and GROOVE. This transformation framework makes it possible to compose transformations while maintaining a formal proof of its syntactical correctness. As discussed in \cref{sec:introduction:related_work}, most active research uses Triple Graph Grammars to deal with the problem of the formalisation of model transformations. This thesis will take a different approach by not modelling EMF/Ecore as a graph language, but rather using a more specific formalisation. Therefore, the formalisation of the transformations will not be based on Triple Graph Grammars, but it will borrow some similar concepts.

Within this work, there will be a focus on the transformations between EMF/Ecore and GROOVE. No earlier work exists that focuses on the formalisation of the transformations between these languages specifically. Because of the focus on these two languages, a practical approach can be used that results in a framework that can be used to create transformations between these two languages directly. Within existing work, either a more abstract method is used, or the formalised transformations are endogenous (e.g., in the work of Biermann, Ermel, and Taentzer \cite{biermann_ermel_taentzer_2011}).

The result of this work can be a valuable foundation for verifying Ecore software models within GROOVE. Furthermore, it could be a valuable contribution to the field of formalised model transformations in general, since it uses an approach different than using TGGs for achieving a formalisation of exogenous transformations.