\section{Related work}
\label{sec:introduction:related_work}
In this section, the work related to this thesis will be discussed. The related work is divided into multiple sections that each describe a different facet related to this thesis.


\subsection{Formalisations of modelling languages}
\label{subsec:introduction:related_work:formalisations_of_modelling_languages}
This section discusses some related work in the field of formalisations of modelling languages. The work presented here is relevant to this thesis as the formalisations of Ecore and GROOVE have an essential role throughout this thesis.

In \cite{kleppe_uml-graph-semantics}, Kleppe and Rensink present a straightforward formalisation of UML models using graph theory and graph constraints. Since Ecore is many facets similar to UML, this formalisation provides a reasonable basis for formalising Ecore as well. Such formalisation has an advantage that it is already built upon graph theory, which allows for an easy formalisation of the transformation to other graph languages. Although the work presented does include formalisations for most relevant elements of UML models, it does not have enough expressive power to formalise concepts unique to Ecore. Within this thesis, a formalisation of Ecore is used that is much closer to the Ecore implementation, with enough expressive power to formalise all the relevant concepts.

Within UML, it is possible to describe a model and its constraints using the Object Constraint Language (OCL) \cite{ocl_spec_2014}. Most queries and invariants written in OCL can also be applied to Ecore models. Moreover, EMF has its declarative language EMF-\textsc{IncQuery} \cite{emf_incquery-guindon_2016}, which can handle complex constraints that cannot be expressed using OCL. 

In \cite{semerath-validation_dsl}, Semeráth et al. present a way to formalise EMF/Ecore by expressing a subset of OCL and EMF-\textsc{IncQuery} in first-order logic. Within this work, each Ecore model is expressed as multiple sets of named elements. These elements are constrained by OCL and EMF-\textsc{IncQuery} invariants, expressed in first-order logic. The goal is to use automated reasoners to analyse the models automatically. Because OCL and EMF-\textsc{IncQuery} are more expressive languages than first-order logic, approximations are used where necessary.

The work presented by Semeráth et al. has a particular relation to this thesis since they try to formalise Ecore to be able to perform formal verification on the Ecore models. In a way, this goal is similar to the goal of this thesis, but the approach is different. Instead of formalising Ecore with the goal of verification, formalising Ecore is in this thesis merely a tool for providing a formalisation of model transformations to GROOVE. Verification is achieved through GROOVE, which is developed solely for this purpose.


\subsection{Formalisations of model transformations}
\label{subsec:introduction:related_work:formalisations_of_model_transformations}
This section discusses related work in the field of formalisations of model transformations. Existing work in this field that is relevant is mostly related to the concept of a Triple Graph Grammar (TGG). Whereas a Graph Grammar can be used to describe the evolution of a single graph model, TGGs allow for describing the relation between two graph models and also allow for transforming one kind of model to the other \cite{KW07_ag}. The formal description of model transformations using TGGs is especially relevant to this thesis, as this thesis will also formalise a specific set of model transformations.

In \cite{hermann_ehrig_golas_orejas_2014}, Hermann, Ehrig, Golas, and Orejas approach the problem of formal analysis of model transformations using triple graph grammars. They explain how triple graph grammars can be used to describe model transformations and which problems arise when performing this task. Properties related to the syntactical correctness, functional behaviour and information preservation are discussed.

The work of Hermann, Ehrig, Golas, and Orejas discusses model transformations on a more abstract level than this thesis, by providing mathematical properties and mathematical structures to approach the problem. These structures and properties are not applied to specific modelling languages. In this thesis uses a more practical approach where Ecore models are transformed to GROOVE graph grammars and vice versa. This approach allows for a mathematical specification that is tailored for these modelling languages and can, therefore, discuss specific properties of these languages in detail.

An application of TGGs on the model transformation of Ecore models is shown by \cite{biermann_ermel_taentzer_2011}. In this work, Biermann, Ermel, and Taentzer use TGGs to formalise the behaviour of model transformations between EMF models. This formalisation is done by formalising EMF models as graph grammars first and then using these graph grammars as part of the TGGs for formalising model transformations within EMF. Ermel, Hermann, Gall, and Binanzer later use this work in \cite{ermel-visual_analysis} to create an Eclipse plugin that can describe model transformations between Ecore diagrams visually, including the possibility to edit them.

The work presented by Biermann, Ermel, and Taentzer uses a formalisation of EMF to describe model transformations formally. This formalisation is similar to the work presented by this thesis but focuses on endogenous transformations (transformations between EMF models) instead of exogenous transformations (transformations from Ecore to GROOVE, in case of this thesis).

In \cite{bruintjes_bridging-groove}, Bruintjes has worked on mapping multiple languages to GROOVE and back using an intermediate conceptual model. This intermediate conceptual model can express Ecore diagrams as well, and therefore Bruintjes provides an implementation of model transformations between Ecore and GROOVE. Because the approach of this work focuses on the implementation, the model transformations are not formalised in this work. It is still worth mentioning because it is the only work that has a focus on transformations between Ecore and GROOVE specifically. Moreover, the conceptual model used within this work does not use graph grammars as a basis, which provides more freedom in expressing specific properties of Ecore.

The work presented by Bruintjes uses a similar approach for formalising Ecore models itself. This thesis will a formalisation inspired by this work, which is like the work of Bruintjes not based on graph grammars. It differs from the work of Bruintjes by focusing on the formal foundation rather than the implementation. Moreover, this thesis only focuses on the model transformations between Ecore and GROOVE, rather than multiple languages and GROOVE.