\section{Formalisation of model transformations}
\label{sec:introduction:formalisation_of_model_transformations}
As explained earlier, model transformations are an automated way of modifying and creating models by transforming existing models. Model transformations can be used in a variety of scenarios, from simple modifications within the same domain and language (an endogenous transformation) to conversions between different domains and languages (an exogenous transformation). Furthermore, model transformations can be unidirectional, meaning that a model can only be transformed one way, or bidirectional, meaning that the model can be transformed in both directions. Unidirectional transformations are particularly useful in situations where the output model is meant to be used as a final result, such as code generation. Bidirectional transformations are necessary for situations where the models must be kept consistent. In that case, a change to one model might necessitate a change to the other model, which then can be automated using model transformations.

Since this thesis focuses on model transformations between EMF/Ecore and GROOVE, this thesis focuses on bidirectional exogenous transformations. The transformations between EMF/Ecore and GROOVE are exogenous by definition, since the languages of EMF/Ecore and GROOVE are different, as will be shown later. The bidirectionality of the transformations is beneficial to ensure consistency, which is a useful property to have in software verification.

In order to prove any property on these model transformations, the transformations need to be formalised. The formalisation of a model transformation consists of mathematical definitions and functions that describe the behaviour of the transformation, allowing to mathematically translate an input model to an output model as described by the model transformation. These definitions and functions directly depend on the formalisations of the input and output models themselves, as these are needed to describe the input and output models of the transformations. Because of this dependency, the formalisations of EMF/Ecore and GROOVE must be established as well.

The main disadvantage of the formalisation of model transformations is the direct relationship between the transformation and its input and output language. As a consequence, the formalisation of a model transformation directly depends on the formalisations of its input and output languages. Therefore, it is not possible to give an abstract formalisation for model transformations between different languages. Creating such a formalisation would mean making the formalisations of the input and output languages more abstract. Making these more abstract might result in loss of information, which is undesirable, or an increase in complexity. Within this thesis, this disadvantage was dealt with by only focusing on the model transformations between EMF/Ecore and GROOVE.