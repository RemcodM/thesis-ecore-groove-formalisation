\section{Application}

\begin{frame}{Example application}
    Combine the transformation framework and library of transformations to compose model transformations between Ecore and GROOVE.
    \begin{itemize}
        \item Small contact application, only names and addresses
    \end{itemize}
\end{frame}

\note{
	\begin{itemize}
	    \item Explain that we will see an example of how the transformation framework can be applied.
	    \item The library of transformation is used as the building blocks.
	    \item A small contacts application, only containing names and addresses is discussed.
	\end{itemize}
}

\begin{frame}{Example application}
    \centering
    \begin{tikzpicture} 
    \path
    (-3,4) node[circle,draw,minimum size=10mm,inner sep=0pt](ME) {$E$}
    (-4.5,2) node[circle,draw,minimum size=10mm,inner sep=0pt](MA) {$M_A$}
    (-1.5,2) node[circle,draw,minimum size=10mm,inner sep=0pt](MB) {$M_B$}
    (-3,0) node[circle,draw,minimum size=10mm,inner sep=0pt](MAB) {$M_{AB}$}
    
    (3,4) node[circle,draw,minimum size=10mm,inner sep=0pt](GN) {$N$}
    (1.5,2) node[circle,draw,minimum size=10mm,inner sep=0pt](GA) {$G_A$}
    (4.5,2) node[circle,draw,minimum size=10mm,inner sep=0pt](GB) {$G_B$}
    (3,0) node[circle,draw,minimum size=10mm,inner sep=0pt](GAB) {$G_{AB}$};
    
    \path[]		
    (ME) [-, black, out=240, in=90] edge node[above] {} (MA)
    (ME) [-, black, out=300, in=90] edge node[above] {} (MB)
    
    (MA) [-{Latex[width=5]}, black, out=270, in=90] edge node[above] {} (MAB)
    (MB) [-{Latex[width=5]}, black, out=270, in=90] edge node[above] {} (MAB)
    
    (GN) [-, black, out=240, in=90] edge node[above] {} (GA)
    (GN) [-, black, out=300, in=90] edge node[above] {} (GB)
    
    (GA) [-{Latex[width=5]}, black, out=270, in=90] edge node[above] {} (GAB)
    (GB) [-{Latex[width=5]}, black, out=270, in=90] edge node[above] {} (GAB)
    
    (ME) [-{Latex[width=5]}, black, out=25, in=155] edge node[above] {$f$} (GN)
    (GN) [-{Latex[width=5]}, black, out=165, in=15] edge node[above] {} (ME)
    
    (MA) [-{Latex[width=5]}, black, out=35, in=145] edge node[above] {$f_A$} (GA)
    (GA) [-{Latex[width=5]}, black, out=155, in=25] edge node[above] {} (MA)
    
    (MB) [-{Latex[width=5]}, black, out=35, in=145] edge node[above] {$f_B$} (GB)
    (GB) [-{Latex[width=5]}, black, out=155, in=25] edge node[above] {} (MB)
    
    (MAB) [-{Latex[width=5]}, black, out=25, in=155] edge node[above] {$f_{A} \sqcup f_{B}$} (GAB)
    (GAB) [-{Latex[width=5]}, black, out=165, in=15] edge node[above] {} (MAB)
    ;
    \end{tikzpicture}
\end{frame}

\note{
	\begin{itemize}
	    \item Explain how the steps will be repeated.
	\end{itemize}
}

\newcommand\typeframe[3]{
\begin{frame}{Step #3/6: Type level}
    \vspace{-0.55cm}
    \begin{columns}[c]
        \begin{column}{0.06\textwidth}
        \end{column}\begin{column}{0.3\textwidth}
            \centering
            \includegraphics[width=0.66\textwidth]{images/05_application/type_model/step#1.pdf}
        \end{column}\begin{column}{0.015\textwidth}
            \centering
            +
        \end{column}\begin{column}{0.3\textwidth}
            \centering
            \includegraphics[width=0.66\textwidth]{images/05_application/type_model/step#2.pdf}
        \end{column}\begin{column}{0.015\textwidth}
            \centering
            =
        \end{column}\begin{column}{0.3\textwidth}
            \centering
            \includegraphics[width=0.66\textwidth]{images/05_application/type_model/step#3.pdf}
        \end{column}
    \end{columns}
    \begin{columns}[c]
        \begin{column}{0.06\textwidth}
        \end{column}\begin{column}{0.3\textwidth}
            \centering
            \rotatebox{90}{$\leftrightarrow$}
        \end{column}\begin{column}{0.015\textwidth}
            \centering
            $\sqcup$
        \end{column}\begin{column}{0.3\textwidth}
            \centering
            \rotatebox{90}{$\leftrightarrow$}
        \end{column}\begin{column}{0.015\textwidth}
            \centering
            =
        \end{column}\begin{column}{0.3\textwidth}
            \centering
            \rotatebox{90}{$\leftrightarrow$}
        \end{column}
    \end{columns} 
    \begin{columns}[c]
        \begin{column}{0.06\textwidth}
        \end{column}\begin{column}{0.3\textwidth}
            \centering
            \input{images/05_application/type_graph/step#1.tikz}
        \end{column}\begin{column}{0.015\textwidth}
            \centering
            +
        \end{column}\begin{column}{0.3\textwidth}
            \centering
            \input{images/05_application/type_graph/step#2.tikz}
        \end{column}\begin{column}{0.015\textwidth}
            \centering
            =
        \end{column}\begin{column}{0.3\textwidth}
            \centering
            \input{images/05_application/type_graph/step#3.tikz}
        \end{column}
    \end{columns} 
\end{frame}
}

\newcommand\instanceframe[3]{
\begin{frame}{Step #3/6: Instance level}
    \vspace{-0.55cm}
    \begin{columns}[c]
        \begin{column}{0.06\textwidth}
        \end{column}\begin{column}{0.2\textwidth}
            \centering
            \includegraphics[width=\textwidth]{images/05_application/instance_model/step#1.pdf}
        \end{column}\begin{column}{0.015\textwidth}
            \centering
            +
        \end{column}\begin{column}{0.35\textwidth}
            \centering
            \includegraphics[width=\textwidth]{images/05_application/instance_model/step#2.pdf}
        \end{column}\begin{column}{0.015\textwidth}
            \centering
            =
        \end{column}\begin{column}{0.35\textwidth}
            \centering
            \includegraphics[width=\textwidth]{images/05_application/instance_model/step#3.pdf}
        \end{column}
    \end{columns}
    \begin{columns}[c]
        \begin{column}{0.06\textwidth}
        \end{column}\begin{column}{0.2\textwidth}
            \centering
            \rotatebox{90}{$\leftrightarrow$}
        \end{column}\begin{column}{0.015\textwidth}
            \centering
            $\sqcup$
        \end{column}\begin{column}{0.35\textwidth}
            \centering
            \rotatebox{90}{$\leftrightarrow$}
        \end{column}\begin{column}{0.015\textwidth}
            \centering
            =
        \end{column}\begin{column}{0.35\textwidth}
            \centering
            \rotatebox{90}{$\leftrightarrow$}
        \end{column}
    \end{columns} 
    \begin{columns}[c]
        \begin{column}{0.06\textwidth}
        \end{column}\begin{column}{0.2\textwidth}
            \centering
            \input{images/05_application/instance_graph/step#1_small.tikz}
        \end{column}\begin{column}{0.015\textwidth}
            \centering
            +
        \end{column}\begin{column}{0.35\textwidth}
            \centering
            \input{images/05_application/instance_graph/step#2.tikz}
        \end{column}\begin{column}{0.015\textwidth}
            \centering
            =
        \end{column}\begin{column}{0.35\textwidth}
            \centering
            \input{images/05_application/instance_graph/step#3.tikz}
        \end{column}
    \end{columns} 
\end{frame}
}

\typeframe{0}{0to1}{1}
\instanceframe{0}{0to1}{1}

\note{
	\begin{itemize}
	    \item Explain how we start with the empty model/graph.
	    \item We add an simple type \textit{Contact}.
	    \item On the instance level, all instances of \textit{Contact} are immediately instantiated.
	\end{itemize}
}

\typeframe{1}{1to2}{2}
\instanceframe{1}{1to2}{2}

\note{
	\begin{itemize}
	    \item An contact is enriched with a name field.
	    \item On the instance level, the field needs to be instantiated for all \textit{Contact}s within the same transformation step.
	\end{itemize}
}

\typeframe{2}{2to3}{3}
\instanceframe{2}{2to3}{3}

\note{
	\begin{itemize}
	    \item Now add a type for \textit{Address}es.
	    \item On the instance level, instantiate once more all instances of \textit{Address}.
	\end{itemize}
}

\typeframe{3}{3to4}{4}
\instanceframe{3}{3to4}{4}

\note{
	\begin{itemize}
	    \item An \textit{Address} is enriched with an address line.
	    \item Like the previous example, all \textit{Address} lines need to be instantiated directly.
	\end{itemize}
}

\typeframe{4}{4to5}{5}
\instanceframe{4}{4to5}{5}

\note{
	\begin{itemize}
	    \item Repeat the previous step with a different field name to add a second field to \textit{Address}, which represented the postal code.
	    \item Like the previous example, all \textit{Address} lines need to be instantiated directly.
	\end{itemize}
}

\typeframe{5}{5to6}{6}
\instanceframe{5}{5to6}{6}

\note{
	\begin{itemize}
	    \item Finally, add a relation between \textit{Contact} and \textit{Address} by referencing the existing types and add a field relation between them.
	    \item On the instance level, the relation needs to be instantiated directly.
	\end{itemize}
}