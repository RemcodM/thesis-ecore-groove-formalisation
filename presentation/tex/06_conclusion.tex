\section{Conclusion}

\begin{frame}{Summary}
    \begin{itemize}
        \item Formalisations for Ecore models and GROOVE graphs.
        \item Formalisations for model transformations between these two languages.
        \item A transformation framework in which transformations can be combined.
        \item A small library of transformations including application.
    \end{itemize}
\end{frame}

\note{
	\begin{itemize}
	    \item Summarise the work this thesis presented shortly.
	\end{itemize}
}

\begin{frame}{Advantages \& Limitations}
    \begin{itemize}
        \item Transformation framework
        \item Incomplete library of transformations
        \item Syntactical correctness only
    \end{itemize}
\end{frame}

\note{
	\begin{itemize}
	    \item Mention that the transformation framework presented has advantages and limitations with respect to maintaining correctness at every step.
	    \item Mention that the library of transformations is still incomplete and additions are needed. Limitation: Is it possible to achieve coverage?
	    \item Mention one more time that this work is syntactical correctness only. There is no guarantee that semantics are preserved.
	\end{itemize}
}

\begin{frame}{Evaluation}
    What is a suitable formalisation for composable model transformations between Ecore and GROOVE that gives rise to correct model transformations between Ecore and GROOVE?
    \pause
    \begin{itemize}
        \item What is a suitable formalisation of Ecore models and what Ecore models are valid within this formalisation? 
        \item What is a suitable formalisation of GROOVE grammars and what GROOVE grammars are valid within this formalisation?
        \item What is a suitable formalisation for the model transformations between Ecore and GROOVE?
        \item What model transformations are correct within the formalisation?
        \item How can correct model transformations between Ecore and GROOVE be composed?
    \end{itemize}
\end{frame}

\note{
	\begin{itemize}
	    \item Mention that the transformation framework presented has advantages and limitations with respect to maintaining correctness at every step.
	    \item Mention that the library of transformations is still incomplete and additions are needed. Limitation: Is it possible to achieve coverage?
	    \item Mention one more time that this work is syntactical correctness only. There is no guarantee that semantics are preserved.
	\end{itemize}
}

\begin{frame}{Future work}
    \begin{itemize}
        \item Improvements to the transformation framework
        \item Complete the library of transformations
        \item Add more encodings
        \item Implementation
    \end{itemize}
\end{frame}

\note{
	\begin{itemize}
	    \item Explain that improvements can be made to the transformation framework.
	    \item Explain that the library of transformations could be completed to obtain complete coverage. Also, it then becomes possible to add alternative encodings.
	    \item Explain that it could be implemented in software. Maybe mention decomposition?
	\end{itemize}
}

